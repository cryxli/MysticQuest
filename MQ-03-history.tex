%% MQ-03-history.tex
\chapter{History}

Since the beginning of the 1980s studios like \textit{Enix}, \textit{Square} and \textit{Ancient} were hard at work to define the genre known today as Japanese Role Playing Game (JRPG) with franchises like \textit{Dragon Quest}, \textit{Final Fantasy} and \textit{Act Racer}.

But by the end of the 1980s sales of games of the RPG genre in the largest market outside of Japan, North America, were still underwhelming. At the time most US gamers were more into sports games.

\textit{Square} decided to write a role-playing game especially for beginners with a western appeal and thus began working on \textit{Mystic Quest} for the \textit{Super Nintendo Entertainment System} (SNES) or \textit{Super Famicom} (SFC) respectively. It was eventually released in North America on Octobre 5, 1992 as \textbf{Final Fantasy Mystic Quest}, in Japan on September 10, 1993 as \textbf{Final Fantasy USA: Mystic Quest} and in Europe and other PAL regions in Octobre 1993 as \textbf{Mystic Quest Legends}. 

Unfortunately \textit{Square} and \textit{Nintendo of America} marketed the game as the next step in the \textit{Final Fantasy} franchise. This kind of miscommunication or lack of understanding of games would plague \textit{Nintendo of America} for years. Although not a huge success in the US, the \textit{Final Fantasy} franchise was well recognised and expectations were high as to what the next \textit{Final Fantasy} game would deliver. 

But \textit{Mystic Quest} is a scaled down, simplified JRPG that lacks most of the depth of previous Final Fantasy titles. Hard-core fans were disappointed. More so, they were worried about future installments of the series. \textit{Mystic Quest} did not open the western markets for \textit{Square} or any other JRPG focused studios. This should only change with \textbf{Final Fantasy VII} five year later, which was a worldwide success.

Despite the game failing the main goal it was set to achieve, reviews were good. Not superbe, but not bad either. Especially the music composed by \textit{Ryuji Sasai} and \textit{Yasuhiro Kawakami} was praised. To this day the score remains one of the best on the \textit{Super Nintendo} and \textit{Super Famicom} system.

The game itself isn’t bad. It’s simpler than \textit{Dragon Quest} or \textit{Final Fantasy} titles of the time, but it is still not trivial. Exploring is as exciting or frustrating as with any JRPG and the absence of random encounters is a nice change in pace. Like early RPGs the game selects the best equipment for the player. But unlike early games, it lets you swap weapons at any time.

Overall \textit{Mystic Quest} is a good and fun game. It is not without flaws. And it's definitely not a \textit{Final Fantasy} game. If you approach it with \textit{Final Fantasy} in mind, you will be disappointed. However, you can still get much joy out of it, if you take it for what it is: A nice little tribute to the first decade of JPRGs.
